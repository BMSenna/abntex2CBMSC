\chapter{A Classe \texttt{abntex2CBMSC}}

A classe \texttt{abntex2CBMSC} foi projetada para implementar os elementos básicos de uma monografia acadêmica, seguindo rigorosamente as normas impostas pela \acrfull{ABNT} para a formatação de produções científicas. A classe oferece uma estrutura organizada e eficiente para a elaboração de trabalhos acadêmicos, considerando as diretrizes da \acrshort{ABNT}, como formatação de margens, espaçamento, tipografia, citações, referências bibliográficas, entre outros.

Com a \texttt{abntex2CBMSC}, os usuários podem se concentrar no conteúdo do trabalho, enquanto a classe cuida da formatação e organização do documento. A classe inclui diversos comandos e pacotes personalizados, como o \texttt{abntex2} e o \texttt{biber}, para automatizar a organização das citações e referências, além de garantir que os elementos do documento estejam em conformidade com as normas brasileiras.

\section{Vantagens do LaTeX}

O LaTeX é uma ferramenta poderosa e flexível para a produção de documentos científicos e técnicos. Entre as principais vantagens do LaTeX, destacam-se:

\begin{itemize}
    \item \textbf{Qualidade tipográfica}: O LaTeX oferece uma qualidade tipográfica superior, proporcionando uma formatação automática e eficiente, que segue as melhores práticas de diagramação.
    
    \item \textbf{Controle preciso sobre a formatação}: A linguagem de marcação do LaTeX permite um controle preciso sobre os detalhes da formatação, como o espaçamento entre linhas, a definição de margens e o alinhamento do texto, tudo de maneira consistente.
    
    \item \textbf{Automação da organização de citações e referências}: Com o LaTeX, a inserção de citações e referências bibliográficas é automatizada, garantindo que elas estejam no formato correto e que sua numeração e apresentação sejam consistentes ao longo do documento.
    
    \item \textbf{Consistência e padronização}: A utilização do LaTeX permite que o usuário se concentre no conteúdo e deixe para o sistema a tarefa de garantir a padronização e a consistência do documento, de acordo com as normas da ABNT ou qualquer outro estilo desejado.
    
    \item \textbf{Portabilidade}: O LaTeX é uma ferramenta independente de plataforma, ou seja, um documento LaTeX pode ser compilado em diferentes sistemas operacionais sem problemas de compatibilidade.
    
    \item \textbf{Estabilidade e qualidade em documentos grandes}: O LaTeX é ideal para documentos grandes, como monografias e teses, permitindo uma organização eficiente e a manutenção de um alto padrão de qualidade na formatação de capítulos, seções, tabelas e figuras.
\end{itemize}

Portanto, a utilização do LaTeX, aliado à classe \texttt{abntex2CBMSC}, oferece aos autores uma forma eficiente e profissional de criar documentos acadêmicos, atendendo às exigências de qualidade e normativas estabelecidas pela \acrshort{ABNT}.
