\chapter{Introdução}

A classe \texttt{abntex2CBMSC} foi desenvolvida com o objetivo de fornecer uma estrutura padronizada para a formatação de monografias acadêmicas, elaboradas como pré-requisitos para a conclusão dos cursos de formação de oficiais, cursos de comando e estado maior, bem como cursos de altos estudos estratégicos da \acrfull{CBMSC}. Esta classe visa atender às especificações e exigências normativas da instituição, promovendo a uniformidade e a clareza na apresentação de trabalhos acadêmicos.

A padronização dos documentos é um aspecto fundamental para garantir a qualidade e a integridade da produção acadêmica, além de facilitar a leitura e a análise crítica dos mesmos. Por meio desta classe, o \acrshort{CBMSC} assegura que todos os trabalhos submetidos sigam um conjunto de diretrizes, oferecendo um formato consistente para os diferentes tipos de documentos gerados pelos alunos de seus diversos cursos.

O presente manual busca apresentar as principais funcionalidades e características da classe \texttt{abntex2CBMSC}, bem como oferecer orientações sobre como utilizá-la corretamente. Assim, espera-se que os usuários possam produzir seus trabalhos acadêmicos de forma eficiente e dentro dos padrões exigidos pela instituição.
