%% ==================================================================
%% Documento baseado na classe abntex2CBMSC.cls
%% ==================================================================
%%
%% Este documento usa a classe abntex2CBMSC, que é uma adaptação da
%% classe abntex2.cls. A adaptação foi realizada para atender às
%% necessidades do Corpo de Bombeiros Militar de Santa Catarina (CBMSC).
%%
%% Autor da adaptação: Jonatas Senna <jonatas.senna.bmsc@gmail.com.com>
%% Instituição: CBMSC
%% Data: 06/02/2025
%%
%% Licença da classe: LaTeX Project Public License (LPPL) v1.3c ou posterior
%% Consulte: https://www.latex-project.org/lppl.txt
%%
%% Este documento pode ser modificado livremente para atender às 
%% necessidades dos usuários, desde que respeitada a licença da classe.
%%
%% ==================================================================

% ------------------------------------------------------------------------
% ------------------------------------------------------------------------
% abnTeX2CBMSC: trabalhos monograficos em geral em conformidade com 
% ABNT NBR 14724:2011: Informacao e documentacao - Trabalhos academicos
% ------------------------------------------------------------------------
% ------------------------------------------------------------------------
% ---
% Classe do Documento, não alterar nada
% ---
\documentclass[
	% -- opções da classe memoir --
	12pt,				% tamanho da fonte
	oneside,			% para impressão em recto e verso. Oposto a oneside
	a4paper,			% tamanho do papel. 
	% -- opções da classe abntex2 --
	chapter = TITLE,		% títulos de capítulos convertidos em letras maiúsculas
	section=TITLE,		% títulos de seções convertidos em letras maiúsculas
	subsection=TITLE,	% títulos de subseções convertidos em letras maiúsculas
	subsubsection=TITLE,% títulos de subsubseções convertidos em letras maiúsculas
	% -- opções do pacote babel --
	english,			% idioma adicional para hifenização
	brazil				% o último idioma é o principal do documento
	]{SRC/abntex2CBMSC}


% ---
% Informações de dados básicos
% ---
\curso{CFO} % CFO, CCEM ou CAEE
\titulo{Uma Classe Latex para Trabalhos de Conclusão de Curso do CBMSC} %Título do trabalho
\autor{Jonatas Ribeiro Senna Pires} %Nome do autor
\local{Florianópolis} %Cidade
\ano{2025} % Ano de apresentação do trabalho
\mes{Fevereiro} % Mês da apresentação do Trabalho
\dia{28} % Dia da apresentação do Trabalho
\instituicao{Corpo de Bombeiros Militar de Santa Catarina} %nome da instituição
\linhaDePesquisa{Editar a linha de pesquisa} % linha de pesquisa segundo a ig40
% ---
% Informações da banca
% até 5 membros no total está bem configurado
% ---
\orientador[\posto{CAP} \corporacao{BM}]{Bruno Lazarin Koch}{CBMSC} % \orientador[título]{Nome}{corporação}
\coorientador[\posto{CAP} \corporacao{BM}]{Diego Medeiros Franz}{CBMSC}%\coorientador[título]{Nome}{corporação}
\adicionarMembro{\posto{TC} \corporacao{BM}}{Juliana Kretzer}{CBMSC} %membro da banca 
\adicionarMembro{\posto{MAJ} \corporacao{BM}}{Natália Cauduro}{CBMSC}
\adicionarMembro{\posto{1TEN} \corporacao{BM}}{Marco Timmermann}{CBMSC}
% ---
% Palavras Chave
% ---
\adicionarPalavraChave{Latex} %palavras chave em português
\adicionarPalavraChave{Monografia}
\adicionarPalavraChave{Padronização}
\adicionarKeyword{Latex} %palavras chave em inglês
\adicionarKeyword{Monograph}
\adicionarKeyword{Standartization}



% ---
% arquivo com definições gerais, pacotes e funções que geram os elementos pré textuais
% ---
\input{SRC/src}


% ----------------------------------------------------------
% ELEMENTOS TEXTUAIS
% ----------------------------------------------------------


% ----------------------------------------------------------
% Introdução
% ----------------------------------------------------------

\input{Texto/introducao}

% ----------------------------------------------------------
% Importa os capítulos de forma automática quando o arquivo é compilado pelo comando fornecido
% ----------------------------------------------------------

\input{capitulos.tex}

% ----------------------------------------------------------
% Conclusão
% ----------------------------------------------------------
\input{Texto/conclusao}

% ----------------------------------------------------------
% Finaliza a parte no bookmark do PDF
% para que se inicie o bookmark na raiz
% e adiciona espaço de parte no Sumário
% ----------------------------------------------------------
\phantompart


% ----------------------------------------------------------
% ELEMENTOS PÓS-TEXTUAIS
% ----------------------------------------------------------
\postextual
% ----------------------------------------------------------

% ----------------------------------------------------------
% Referências bibliográficas
% ----------------------------------------------------------
% \bibliography{abntex2-modelo-references}

\newpage

\defbibheading{bibliography}[\refname]{%
  \section*{\centering\textbf{\MakeUppercase{#1}}} % Centraliza, textbf e maiúsculo
}
% Aumentando o espaçamento entre itens da bibliografia
\setlength{\bibitemsep}{0.6\baselineskip}
\printbibliography


% ----------------------------------------------------------
% Apêndices
% ----------------------------------------------------------

% ---
% Inicia os apêndices
% ---
\imprimirapendices
% ---


% ----------------------------------------------------------
% Anexos
% ----------------------------------------------------------

% ---
% Inicia os anexos
% ---
\imprimiranexos

%---------------------------------------------------------------------
% INDICE REMISSIVO
%---------------------------------------------------------------------
\phantompart
\printindex
%---------------------------------------------------------------------

\end{document}
